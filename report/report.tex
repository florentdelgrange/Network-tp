\documentclass[10pt,a4paper]{article}
\usepackage[utf8]{inputenc}
\usepackage[francais]{babel}
\usepackage[T1]{fontenc}
\usepackage{listings}
\usepackage{amsmath}
\usepackage{amsfonts}
\usepackage{amssymb}
\usepackage{graphicx}
\author{Gheysen Jeremy, Delgrange Florent}
\title{Réseaux II : TP 1}
\begin{document}
\maketitle
\noindent Connection au router :
\begin{lstlisting}[language=bash]
 telnet 192.168.254.144 9004
\end{lstlisting}

\noindent configuration:
\begin{lstlisting}[language=bash]
 enable # acces to more privileges
 conf t # configuration mode; press ctrl z to quit 
\end{lstlisting}

\noindent Connection au router :
\begin{lstlisting}[language=bash]
 sh int # up or down informations
 sh ip route # show interface route ip attach with the network
\end{lstlisting}

\textit{sh int} nous indique que la connexion est établie mais qu'aucune interface n'est encore créée (parce que pas encore configurée) et que celui qui est connecté au $5$ est up (car notre router est connecté au $5^{eme}$).

\noindent Ouvrir les ports ethernets (être en \textbf{conf t})
\begin{lstlisting}[language=bash]
int ethernet 0/0 # we acces to ethernet 0/0 port
no shutdown # up the ethernet port
\end{lstlisting}

On connecte maintenant le rooter au switch sur l'interface ethernet 1/1. On attribue maintenant une adresse à l'interface du router.

\noindent Attribuer une adresse IP à l'interface.
\begin{lstlisting}[language=bash]
conf t
int ethernet 1/1
ip address 192.168.104.002 255.255.255.248
\end{lstlisting}
\textbf{Pourquoi 3 bits ?}\\
Le 1er bit réservé (1 pour le broadcast 0 pour la machine). On a 4 ports de disponibles.\\
On configure par la suite toutes les interfaces de cette façon.

\noindent Configurer le DHCP
\begin{lstlisting}[language=bash]
ip dhcp pool giorgio
network 192.168.104.0 255.255.255.248
dns-server 192.168.104.0
default-router 192.168.104.0
lease 5 0 0
\end{lstlisting}

En tentant de connecter un PC, celui-ci obtient bien une adresse depuis le router, on peut la récupérer grace à la commande \textbf{ifconfig} : 192.168.104.7 \\
Si on ping le routeur, celui-ci répond également positivement avec 

\noindent ping
\begin{lstlisting}[language=bash]
ping 192.168.104.2
PING 192.168.104.2 (192.168.104.2): 56 data bytes
64 bytes from 192.168.104.2: icmp_seq=0 ttl=255 time=2.050 ms
64 bytes from 192.168.104.2: icmp_seq=2 ttl=255 time=2.076 ms
64 bytes from 192.168.104.2: icmp_seq=3 ttl=255 time=2.062 ms
64 bytes from 192.168.104.2: icmp_seq=4 ttl=255 time=2.311 ms
...
\end{lstlisting}
show ip DHCP pool

\noindent Connexion avec le routeur 5
\begin{lstlisting}[language=bash]
int ethernet 0/1
ip adress 192.168.142.2 255.255.255.248
\end{lstlisting}

\noindent Configuration OSPF
\begin{lstlisting}[language=bash]
conf t
router ospf 10
newtork 192.168.104.0 0.0.0.255 area 0
\end{lstlisting}
\end{document}